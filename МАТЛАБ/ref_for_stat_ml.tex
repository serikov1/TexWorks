\documentclass[a4paper, 12pt]{article}%тип документа

%отступы
\usepackage[left=3cm,right=1.5cm,top=2cm,bottom=2cm,bindingoffset=0cm]{geometry}
\setlength{\parindent}{5ex}

%Русский язык

\usepackage[T2A]{fontenc} %кодировка
\usepackage[utf8]{inputenc} %кодировка исходного кода
\usepackage[english,russian]{babel} %локализация и переносы
\usepackage{fontspec}
\setmainfont[Ligatures={TeX,Historic}]{Times New Roman}
%Вставка картинок
\usepackage{graphicx}
\graphicspath{{pictures/}}
\DeclareGraphicsExtensions{.pdf,.png,.jpg,.bmp,}
\usepackage{wrapfig}

%Графики
\usepackage{pgfplots}
\pgfplotsset{compat=1.9}

%Математика
\usepackage{amsmath, amsfonts, amssymb, amsthm, mathtools}
\usepackage{listings}

%\usepackage[usenames,dvipsnames]{color}
\usepackage[numbered,framed]{mcode}


%Таблицы
\usepackage{longtable} 
\usepackage{float}

%Римские цифры
\newcommand{\RomanNumeralCaps}[1]{\uppercase\expandafter{\romannumeral#1}}

\usepackage{multirow}

\setcounter{tocdepth}{2}


\begin{document}
	\begin{titlepage}
		\begin{center}
			\textsc{Федеральное государственное автономное образовательное учреждение высшего образования«Московский физико-технический институт (национальный исследовательский университет)»\\[5mm]
			}
			
			\vfill
			
			\textbf{Проектная работа: \\[3mm]
				Справка по использованию statistics and machine learning toolbox
				\\[50mm]
			}
			
		\end{center}
		
		\hfill
		\begin{minipage}{.5\textwidth}
			Выполнил студент:\\[2mm]
			Сериков Василий Романович\\[2mm]
			Группа: Б03-102\\[5mm]
			
		\end{minipage}
		\vfill
		\begin{center}
			Москва, 2024 г.
		\end{center}
		
	\setcounter{page}{1}	
	\newpage
	\tableofcontents
	\newpage
	\section{Введение}
	\hspace{\parindent} Statistics and Machine Learning Toolbox (SMLT) в MATLAB предоставляет широкий спектр функций для обработки экспериментальных данных и постановки эксперимента. Эта справка описывает основные функции SMLT и их применение для решения типичных задач.
	SMLT является одним из наиболее популярных и мощных инструментов для обработки экспериментальных данных, постановки эксперимента и машинного обучения в MATLAB.\\
	
	\textbf{История развития:}\\
	SMLT был впервые представлен в MATLAB в 1993 году под названием Statistics Toolbox. В течение многих лет инструмент постоянно развивался и совершенствовался, добавлялись новые функции и возможности. В 2012 году инструмент был переименован в Statistics and Machine Learning Toolbox, чтобы отразить его расширенные возможности в области машинного обучения.\\
	
	\textbf{Состав инструмента:}\\
	SMLT состоит из множества функций и инструментов для обработки экспериментальных данных, постановки эксперимента и машинного обучения. Инструмент включает в себя следующие компоненты:\\
	1) Функции для описательной статистики, такие как вычисление среднего значения, дисперсии, медианы, квантилей и других характеристик.\\
	2) Функции для визуализации данных, такие как построение гистограмм, бокс-плотов, диаграмм рассеяния и других графиков.\\
	3) Функции для статистического вывода, такие как проверка гипотез, построение доверительных интервалов, регрессионный анализ и другие методы.\\
	4) Функции для машинного обучения, такие как обучение классификаторов, регрессионных моделей, моделей кластеризации и других алгоритмов.\\
	5) Функции для обработки временных рядов, такие как анализ спектра, фильтрация, прогнозирование и другие методы.\\
	6) Функции для оптимизации, такие как линейное и нелинейное программирование, оптимизация по критерию максимального правдоподобия и другие методы.\\
	
	
	\textbf{Преимущества инструмента:}\\
	SMLT имеет множество преимуществ перед другими инструментами для обработки экспериментальных данных, постановки эксперимента и машинного обучения. Некоторые из них:\\
	1) Возможность использования в среде MATLAB, которая предоставляет широкие возможности для математического моделирования, визуализации данных и других вычислений.\\
	2) Наличие большого количества готовых функций и инструментов для решения типичных задач в области статистики и машинного обучения.\\
	3) Возможность расширения функциональности инструмента за счет использования дополнительных тулбоксов и библиотек.\\
	4) Наличие подробной документации и примеров кода, которые помогают освоить инструмент и эффективно использовать его возможности.\\
	5) Возможность интеграции с другими инструментами и языками программирования, такими как Python, C++ и другими.\\
	
	\textbf{Функции для анализа данных:}\\
	Среди функций для анализа данных в Statistics and Machine Learning Toolbox можно выделить следующие:\\
	1) Функции для вычисления статистических характеристик, таких как среднее значение, медиана, дисперсия, коэффициент вариации и другие.\\
	2) Функции для проверки гипотез, такие как t-тест, критерий хи-квадрат, критерий Колмогорова-Смирнова и другие.\\
	3) Функции для анализа дисперсии, такие как однофакторный и многофакторный дисперсионный анализ, анализ ковариации и другие.\\
	4) Функции для анализа временных рядов, такие как авторегрессионная модель, модель движущего среднего, модель авторегрессии с движущим средним и другие.\\
	5) Функции для кластерного анализа, такие как иерархический кластерный анализ, k-средних, метод главных компонент и другие.\\
	
	\textbf{Функции для машинного обучения:}\\
	Среди функций для машинного обучения в Statistics and Machine Learning Toolbox можно выделить следующие:\\
	1) Функции для создания и обучения классификаторов, таких как линейные и нелинейные классификаторы, деревья решений, нейронные сети и другие.\\
	2) Функции для создания и обучения регрессионных моделей, таких как линейная регрессия, логарифмическая регрессия, регрессия на основе деревьев и другие.\\
	3) Функции для создания и обучения моделей кластеризации, таких как k-средних, иерархической кластеризации, кластеризации на основе плотности и других методов.\\
	4) Функции для оценки качества обученных моделей, такие как вычисление точности, чувствительности, специфичности, F1-меры и других метрик.\\
	5) Функции для подбора гиперпараметров, такие как поиск сетки, кросс-валидация, байесовская оптимизация и другие методы.\\
	
	\section{Обработка экспериментальных данных}

	\hspace{\parindent} \textbf{Очистка данных:}\\
		Очистка данных - это важный этап подготовки данных для последующего анализа. В SMLT существует несколько функций для очистки данных, одна из которых - \textit{rmmissing}. Эта функция удаляет строки с пропущенными значениями из таблицы. Например, если у нас есть таблица \textit{data} с пропущенными значениями, то мы можем удалить эти строки следующим образом:\\
		
	\begin{lstlisting}
	data = rmmissing(data);
	\end{lstlisting}
	Если же нужно удалить столбцы с пропущенными значениями, то можно использовать опцию \textit{'columns'}:\\
	
	\begin{lstlisting}
	data = rmmissing(data,'columns');
	\end{lstlisting}
	
	\textbf{Статистический анализ:}\\
	\hspace{\parindent}Статистический анализ - это неотъемлемая часть обработки экспериментальных данных. В SMLT существует множество функций для статистического анализа, в том числе \textit{mean}, \textit{std}, \textit{corrcoef} и другие.
	
	Функция \textit{mean} вычисляет среднее арифметическое значение элементов массива. Например, если у нас есть вектор \textit{x}, то среднее арифметическое вычисляется следующим образом:\\
	
	
	\begin{lstlisting}
	m = mean(x);
	\end{lstlisting}
	
	Функция \textit{std} вычисляет стандартное отклонение элементов массива. Например, стандартное отклонение вектора \textit{x} вычисляется следующим образом:\\
	
	\begin{lstlisting}
	s = std(x);
	\end{lstlisting}
	
	Функция \textit{corrcoef} вычисляет матрицу коэффициентов корреляции Пирсона. Например, если у нас есть две переменные \textit{x} и \textit{y}, то матрица коэффициентов корреляции вычисляется следующим образом:\\
	
	\begin{lstlisting}
	R = corrcoef(x,y);
	\end{lstlisting}
	
	\textbf{Визуализация данных:}\\
	Функция \textit{fitrlinear} обучает линейную регрессионную модель, \textit{fitrtree} - модель регрессии на базе дерева решений.


	Визуализация данных - это важный этап анализа экспериментальных данных. В SMLT существует множество функций для визуализации данных, в том числе \textit{histogram}, \textit{boxplot}, \textit{scatter} и другие.
	
	Функция \textit{histogram} строит гистограмму, которая показывает распределение данных. Например, если у нас есть вектор \textit{x}, то гистограмма строится следующим образом:\\
	
	\begin{lstlisting}
	histogram(x);
	\end{lstlisting}
	
	Функция \textit{boxplot} строит бокс-плот, который показывает статистическое распределение данных. Например, если у нас есть вектор \textit{x}, то бокс-плот строится следующим образом:\\
	
	\begin{lstlisting}
	boxplot(x);
	\end{lstlisting}
	
	Функция \textit{scatter} строит точечную диаграмму, которая показывает зависимость между двумя переменными. Например, если у нас есть две переменные \textit{x} и \textit{y}, то точечная диаграмма строится следующим образом:\\
	
	\begin{lstlisting}
	scatter(x,y);
	\end{lstlisting}
	
	\section{Постановка эксперимента}
	
	\hspace{\parindent}\textbf{Планирование эксперимента:}\\
	Планирование эксперимента - это важный этап постановки эксперимента. В SMLT существует несколько функций для планирования эксперимента, одна из которых - \textit{designmatrix}. Эта функция создает матрицу планирования эксперимента. Например, если у нас есть факторы \textit{A}, \textit{B} и \textit{C}, то матрица планирования эксперимента создается следующим образом:\\
	
	\begin{lstlisting}
	D = designmatrix([A,B,C],'full');
	\end{lstlisting}
	
	Функция \textit{rng} инициализирует генератор случайных чисел для создания случайных выборок. Например, если нужно создать случайную выборку размера \textit{n} из нормального распределения с параметрами \textit{mu} и \textit{sigma}, то это можно сделать следующим образом:\\
	
	\begin{lstlisting}
	rng('shuffle');
	x = normrnd(mu,sigma,n,1);
	\end{lstlisting}
	
	\textbf{Анализ результатов эксперимента:}\\
	Анализ результатов эксперимента - это неотъемлемая часть постановки эксперимента. В SMLT существует множество функций для анализа результатов эксперимента, в том числе \textit{anova1}, \textit{ttest2}, \textit{regress} и другие.

	Функция \textit{anova1} проводит однофакторный дисперсионный анализ. Например, если у нас есть переменная \textit{y} и фактор \textit{A}, то однофакторный дисперсионный анализ проводится следующим образом:\\
	
	\begin{lstlisting}
	[p,tbl,stats] = anova1(y,A);
	\end{lstlisting}
	
	Функция \textit{ttest2} проводит двухвыборочный t-тест. Например, если у нас есть две выборки \textit{x} и \textit{y}, то двухвыборочный t-тест проводится следующим образом:\\
	
	\begin{lstlisting}
	[h,p] = ttest2(x,y);
	\end{lstlisting}
	Функция \textit{regress} проводит линейную регрессию. Например, если у нас есть переменная \textit{y} и матрица предикторов \textit{X}, то линейная регрессия проводится следующим образом:\\
	
	\begin{lstlisting}
	mdl = fitlm(X,y);
	\end{lstlisting}

	\textbf{Дополнительные функции:}\\
	Кроме вышеупомянутых функций, в SMLT существует множество других функций для постановки эксперимента. Например, функция \textit{anovan} проводит n-факторный дисперсионный анализ, функция \textit{kruskalwallis} - критерий Краскела-Уоллиса, функция \textit{friedman} - критерий Фридмана, функция \textit{durbinwatson} - статистика Дербина-Уотсона, функция \textit{levene} - критерий Левена, функция \textit{median} вычисляет медиану элементов массива, функция \textit{mode} - моду, функция \textit{quantile} - квантили, функция \textit{interquartile} - межквартильное расстояние, функция \textit{skewness} - коэффициент асимметрии, функция \textit{kurtosis} - коэффициент эксцесса и другие.
	
	\section{Машинное обучение}
	
	\hspace{\parindent} Машинное обучение - это одна из ключевых областей применения SMLT. В этом разделе будут рассмотрены некоторые функции SMLT для машинного обучения, которые были упомянуты ранее.\\
	
	\textbf{Классификация: }\\
	Классификация - это задача машинного обучения, которая заключается в предсказании класса объекта на основе его характеристик. В SMLT существует множество функций для классификации, в том числе \textit{fitcsvm}, \textit{fitctree}, \textit{fitcdiscr} и другие.
	Функция \textit{fitcsvm} обучает модель классификации на базе метода опорных векторов. Например, если у нас есть матрица предикторов \textit{X} и вектор ответов \textit{Y}, то модель классификации обучается следующим образом:\\
	
	
	\begin{lstlisting}
	mdl = fitcsvm(X,Y);
	\end{lstlisting}	
	
	
	Функция \textit{fitctree} обучает модель классификации на базе дерева решений. Например, если у нас есть матрица предикторов \textit{X} и вектор ответов \textit{Y}, то модель классификации обучается следующим образом:\\
	
	\begin{lstlisting}
	mdl = fitctree(X,Y);
	\end{lstlisting}
	
	Функция \textit{fitcdiscr} обучает модель классификации на базе дискриминантного анализа. Например, если у нас есть матрица предикторов \textit{X} и вектор ответов \textit{Y}, то модель классификации обучается следующим образом:\\
	
	
	\begin{lstlisting}
	mdl = fitcdiscr(X,Y);
	\end{lstlisting}
	
	
	\textbf{Регрессия:}\\
	Регрессия - это задача машинного обучения, которая заключается в предсказании числового значения на основе характеристик объекта. В SMLT существует множество функций для регрессии, в том числе \textit{fitrlinear}, \textit{fitrtree}, \textit{fitrensemble} и другие.
	
	Функция \textit{fitrlinear} обучает линейную регрессионную модель. Например, если у нас есть матрица предикторов \textit{X} и вектор ответов \textit{Y}, то линейная регрессионная модель обучается следующим образом:\\
	
	
	\begin{lstlisting}
	mdl = fitrlinear(X,Y);
	\end{lstlisting}
	
	
	Функция \textit{fitrtree} обучает модель регрессии на базе дерева решений. Например, если у нас есть матрица предикторов \textit{X} и вектор ответов \textit{Y}, то модель регрессии обучается следующим образом:\\
	
	
	\begin{lstlisting}
	mdl = fitrtree(X,Y);
	\end{lstlisting}
	
	
	Функция \textit{fitrensemble} обучает ансамблевую модель регрессии. Например, если у нас есть матрица предикторов \textit{X} и вектор ответов \textit{Y}, то ансамблевая модель регрессии обучается следующим образом:\\
	
	
	\begin{lstlisting}
	mdl = fitrensemble(X,Y);
	\end{lstlisting}
	
	
	\textbf{Оценка качества модели:}\\
	Оценка качества модели - это один из важных этапов машинного обучения. В SMLT существует множество функций для оценки качества модели, в том числе \textit{confusionmat}, \textit{roccurve}, \textit{liftcurve} и другие.
	
	Функция \textit{confusionmat} строит матрицу ошибок классификации. Например, если у нас есть модель классификации \textit{mdl} и вектор ответов \textit{Y}, то матрица ошибок классификации строится следующим образом:\\
	
	
	\begin{lstlisting}
	C = confusionmat(Y,predict(mdl,X));
	\end{lstlisting}
	
	
	Функция \textit{roccurve} строит ROC-кривую. Например, если у нас есть модель классификации \textit{mdl} и вектор ответов \textit{Y}, то ROC-кривая строится следующим образом:\\
	
	
	\begin{lstlisting}
	[X,Y,T,AUC] = roccurve(Y,predict(mdl,X));
	\end{lstlisting}
	
	
	Функция \textit{liftcurve} строит кривую лифта. Например, если у нас есть модель классификации \textit{mdl} и вектор ответов \textit{Y}, то кривая лифта строится следующим образом:\\
	
	
	\begin{lstlisting}
	lift = liftcurve(Y,predict(mdl,X));
	\end{lstlisting}
	
	
	\textbf{Дополнительные функции:}\\	
	Кроме вышеупомянутых функций, в SMLT существует множество других функций для машинного обучения. Например, функция \textit{fitcknn} обучает модель классификации на базе k-ближайших соседей, функция \textit{fitcecoc} обучает модель классификации на базе кодирования ошибок, функция \textit{fitcensemble} обучает ансамблевую модель классификации, функция \textit{fitrknn} обучает модель регрессии на базе k-ближайших соседей, функция \textit{fitrensemble} обучает ансамблевую модель регрессии и другие.
	
	\section{Примеры применения SMLT}
	
	В этом разделе будут приведены несколько примеров применения SMLT для решения типичных задач.\\
	
	\textbf{Обработка экспериментальных данных:}\\
	Предположим, что у нас есть экспериментальные данные о зависимости температуры воздуха от времени. Необходимо построить график зависимости и вычислить коэффициент корреляции Пирсона.
	
	Для решения этой задачи можно воспользоваться функциями \textit{plot} и \textit{corrcoef}. Пусть у нас есть векторы \textit{t} и \textit{T}, содержащие значения времени и температуры соответственно. Тогда график зависимости и коэффициент корреляции Пирсона можно получить следующим образом:\\
	
	
	\begin{lstlisting}
	plot(t,T);
	xlabel('Time, s');
	ylabel('Temperature, C');
	R = corrcoef(t,T);
	\end{lstlisting}
	
	
	Рассмотрим еще пример. Предположим, что у нас есть данные о росте растений в зависимости от количества удобрений. Мы хотим провести анализ дисперсии, чтобы определить, влияет ли количество удобрений на рост растений.
	
	Для этого мы можем воспользоваться функцией \textit{anova1}, которая выполняет однофакторный дисперсионный анализ. Данные можно загрузить в MATLAB с помощью функции \textit{readtable}. Затем мы можем выполнить анализ дисперсии и проверить гипотезу о том, что средние значения роста растений в разных группах одинаковы.\\
	
	
	\begin{lstlisting}
	% Load data
	data = readtable('plant_growth.csv');
	
	% Performing one-way ANOVA
	[p,tbl,stats] = anova1(data.Growth, data.Fertilizer);
	
	%  Testing the hypothesis of equality of average values
	alpha = 0.05;
	if p < alpha
		disp('The hypothesis of equality of average values is rejected')
	else
		disp('The hypothesis of equality of average values is accepted')
	end
	\end{lstlisting}
	

	
	\textbf{Постановка эксперимента:}\\
	Предположим, что необходимо провести эксперимент по изучению влияния температуры на прочность материала. Для этого нужно спланировать эксперимент, провести его и проанализировать результаты.
	
	Для решения этой задачи можно воспользоваться функциями \textit{designmatrix}, \textit{rng}, \textit{anova1}. Пусть у нас есть фактор \textit{Temperature} с тремя уровнями (20, 50, 80 градусов Цельсия) и фактор \textit{Material} с двумя уровнями (материал А и материал Б). Тогда матрица планирования эксперимента, случайная выборка и результаты однофакторного дисперсионного анализа могут быть получены следующим образом:\\
	
	
	\begin{lstlisting}
	D = designmatrix([Temperature,Material],'full');
	rng('shuffle');
	% Sample size 
	n = 10;
	% Random sample
	X = D(:,1:2)*[20,50,80;0,0,1]+normrnd(0,5,n,2);
	% Result of experiment
	Y = X*[1;2;3]+normrnd(0,10,n,1);
	% One-way ANOVA
	[p,tbl,stats] = anova1(Y,D(:,3));
	\end{lstlisting}
	
	
	\textbf{Машинное обучение:}\\
	Предположим, что необходимо построить модель классификации для предсказания вида растения по его характеристикам. Для этого нужно обучить модель на обучающей выборке и проверить ее на тестовой выборке.
	
	Для решения этой задачи можно воспользоваться функциями \textit{fitcsvm}, \textit{predict}. Пусть у нас есть матрица предикторов \textit{X} и вектор ответов \textit{Y}, содержащие характеристики растений и их виды соответственно. Тогда модель классификации и ее точность на тестовой выборке могут быть получены следующим образом:\\
	
	
	\begin{lstlisting}
	% Model training
	mdl = fitcsvm(Xtrain,Ytrain); 
	% Prediction on test sample
	Ypred = predict(mdl,Xtest); 
	% Model accuracy
	accuracy = sum(Ypred==Ytest)/length(Ytest); 
	\end{lstlisting}
	
	Рассмотрим еще пример. Предположим, что у нас есть данные о клиентах банка, и мы хотим построить модель классификации, которая будет предсказывать, будет ли клиент брать кредит в банке.
	
	Для этого мы можем воспользоваться функцией \textit{fitcecoc}, которая обучает модель классификации на основе метода ошибок и кодов. Данные можно загрузить в MATLAB с помощью функции \textit{readtable}. Затем мы можем разбить данные на обучающую и тестовую выборки, обучить модель классификации и проверить ее качество на тестовой выборке.\\
	
	
	\begin{lstlisting}
	% Load data
	data = readtable('bank_data.csv');
	
	% Dividing data into training and test samples
	cvp = cvpartition(data.Response,'HoldOut',0.3);
	dataTrain = data(cvp.training,:);
	dataTest = data(cvp.test,:);
	
	% Classification model training
	mdl = fitcecoc(dataTrain,'Response','Learners','logistic');
	
	% Prediction on test sample
	labelsPred = predict(mdl, dataTest);
	
	% Calculating model quality
	confMat = confusionmat(dataTest.Response, labelsPred);
	accuracy = sum(diag(confMat))/sum(confMat(:));
	disp(['Model accuracy: ', num2str(accuracy)])
	\end{lstlisting}
	
	
	\textbf{Использования функций для кластеризации:}\\
	Рассмотрим пример использования функций для кластеризации в Statistics and Machine Learning Toolbox. Предположим, что у нас есть данные о покупателях интернет-магазина, и мы хотим разделить их на группы в зависимости от их поведения при покупке.
	Для этого мы можем воспользоваться функцией \textit{kmeans}, которая выполняет кластеризацию на основе метода k-средних. Данные можно загрузить в MATLAB с помощью функции \textit{readtable}. Затем мы можем выполнить нормализацию данных, выбрать количество кластеров, выполнить кластеризацию и визуализировать результаты.\\
	
	
	\begin{lstlisting}
	% Load data
	data = readtable('online_shoppers.csv');
	
	% Data normalization
	dataNorm = normalize(data{:,2:end});
	
	% Selecting the number of clusters
	numClusters = 3;
	
	% Performing clustering
	[idx,C] = kmeans(dataNorm,numClusters);
	
	% Visualization of results
	figure;
	gscatter(dataNorm(:,1),dataNorm(:,2),idx);
	title('Clustering online store clients');
	xlabel('Time on site (normalized)');
	ylabel('Total cost of purchases (normalized)');
	\end{lstlisting}
	
	
	\textbf{Использования функций для регрессионного анализа:}\\
	Рассмотрим пример использования функций для регрессионного анализа в Statistics and Machine Learning Toolbox. Предположим, что у нас есть данные о ценах на недвижимость в разных районах города, и мы хотим построить модель регрессии, которая будет предсказывать цену на недвижимость в зависимости от площади и количества комнат.
	Для этого мы можем воспользоваться функцией \textit{fitlm}, которая обучает линейную модель регрессии. Данные можно загрузить в MATLAB с помощью функции \textit{readtable}. Затем мы можем разбить данные на обучающую и тестовую выборки, обучить модель регрессии и проверить ее качество на тестовой выборке.\\
	
	
	\begin{lstlisting}
	% Load data
	data = readtable('real_estate_prices.csv');
	
	% Dividing data into training and test samples
	cvp = cvpartition(size(data,1),'HoldOut',0.3);
	dataTrain = data(cvp.training,:);
	dataTest = data(cvp.test,:);
	
	% Training a regression model
	mdl = fitlm(dataTrain,'Price~Area+Rooms');
	
	% Prediction on a test sample
	labelsPred = predict(mdl, dataTest);
	
	% Calculating model quality
	mse = mean((dataTest.Price - labelsPred).^2);
	rmse = sqrt(mse);
	disp(['Standard deviation: ', num2str(rmse)])
	\end{lstlisting}
	
	
	\textbf{Использования функций для глубокого обучения:}\\
	Рассмотрим пример использования функций для глубокого обучения в Statistics and Machine Learning Toolbox. Предположим, что у нас есть набор изображений, и мы хотим обучить нейронную сеть, которая будет классифицировать изображения по их классам.
	
	Для этого мы можем воспользоваться функцией \textit{trainNetwork}, которая обучает нейронную сеть на основе метода обратного распространения ошибки. Данные можно загрузить в MATLAB с помощью функции \textit{imageDatastore}. Затем мы можем разбить данные на обучающую и тестовую выборки, определить архитектуру нейронной сети, обучить ее и проверить ее качество на тестовой выборке.\\
	
	
	\begin{lstlisting}
	% Load data
	imds = imageDatastore('image_dataset','IncludeSubfolders',
						  true,'LabelSource','foldernames');
	
	% Dividing data into training and test samples
	[imdsTrain,imdsTest] = splitEachLabel(imds,0.3);
	
	% Defining a neural network architecture
	layers = [
	imageInputLayer([28 28 1])
	convolution2dLayer(3,8,'Padding','same')
	reluLayer
	maxPooling2dLayer(2,'Stride',2)
	fullyConnectedLayer(10)
	softmaxLayer
	classificationLayer];
	
	% Neural network training
	options = trainingOptions('sgdm', ...
	'MaxEpochs',20, ...
	'MiniBatchSize',128, ...
	'ValidationData',imdsTest, ...
	'ValidationFrequency',500, ...
	'Verbose',false, ...
	'Plots','training-progress');
	net = trainNetwork(imdsTrain,layers,options);
	
	% Prediction on a test sample
	labelsPred = classify(net,imdsTest);
	
	% Calculating model quality
	confMat = confusionmat(imdsTest.Labels,labelsPred);
	accuracy = sum(diag(confMat))/sum(confMat(:));
	disp(['Model accuracy: ', num2str(accuracy)])
	\end{lstlisting}
	
	
	\section{Дополнительные возможности}
	
	SMLT предоставляет множество дополнительных возможностей для обработки экспериментальных данных, постановки эксперимента и машинного обучения. В этом разделе будут рассмотрены некоторые из них.\\
	
	\textbf{Обработка временных рядов:}\\
	SMLT предоставляет множество функций для обработки временных рядов. Например, функция \textit{diff} вычисляет разность между соседними элементами временного ряда, функция \textit{detrend} удаляет трендовую компоненту из временного ряда, функция \textit{seasonal} выделяет сезонную компоненту из временного ряда, функция \textit{arima} обучает модель ARIMA для прогнозирования временного ряда и другие.
	
	\textbf{Анализ частот:}\\
	SMLT предоставляет множество функций для анализа частот. Например, функция \textit{fft} вычисляет быстрое преобразование Фурье, функция \textit{periodogram} вычисляет периодограмму, функция \textit{spectrogram} вычисляет спектрограмму, функция \textit{coherence} вычисляет когерентность между двумя сигналами и другие.
	
	\textbf{Обработка изображений:}\\
	SMLT предоставляет множество функций для обработки изображений. Например, функция \textit{imread} читает изображение с диска, функция \textit{imresize} изменяет размер изображения, функция \textit{imfilter} фильтрует изображение, функция \textit{imshow} отображает изображение, функция \textit{imhist} вычисляет гистограмму изображения и другие.
	
	\textbf{Обработка сигналов:}\\
	SMLT предоставляет множество функций для обработки сигналов. Например, функция \textit{filter} фильтрует сигнал, функция \textit{conv} вычисляет свертку сигналов, функция \textit{corr} вычисляет корреляцию сигналов, функция \textit{spectrum} вычисляет спектр сигнала, функция \textit{findpeaks} находит пики сигнала и другие.
	
	\textbf{Оптимизация:}\\
	SMLT предоставляет множество функций для оптимизации. Например, функция \textit{fminunc} минимизирует функцию с помощью метода квазиньютоновской оптимизации, функция \textit{fmincon} минимизирует функцию с ограничениями, функция \textit{fminsearch} минимизирует функцию с помощью метода Нелдера-Мида, функция \textit{patternsearch} минимизирует функцию с помощью метода поиска по шаблону и другие.
	
	\textbf{Статистический контроль качества:}\\
	SMLT предоставляет множество функций для статистического контроля качества. Например, функция \textit{controlchart} строит график управления качеством, функция \textit{capability} вычисляет способность процесса, функция \textit{pchart} строит p-график, функция \textit{npchart} строит np-график, функция \textit{uchart} строит u-график и другие.
	
	\textbf{Многомерная статистика:}\\
	SMLT предоставляет множество функций для многомерной статистики. Например, функция \textit{pca} вычисляет главные компоненты, функция \textit{mdscale} вычисляет многомерное масштабирование, функция \textit{cluster} выполняет кластеризацию, функция \textit{discriminant} вычисляет дискриминантный анализ, функция \textit{canoncorr} вычисляет каноническую корреляцию и другие.
	
	\section{Рекомендации по использованию SMLT}
	
	При использовании SMLT необходимо соблюдать некоторые рекомендации, чтобы эффективно решать задачи обработки экспериментальных данных, постановки эксперимента и машинного обучения.
	
	\textbf{Подготовка данных:}\\
	Перед использованием функций SMLT необходимо подготовить данные. Это включает в себя очистку данных от ошибок и пропусков, нормализацию и стандартизацию данных, выбор подходящих предикторов и ответов, разбиение данных на обучающую и тестовую выборки и другие.
	
	\textbf{Выбор подходящих функций:}\\
	SMLT предоставляет множество функций для решения различных задач. Необходимо выбирать подходящие функции в зависимости от типа данных, цели анализа и других факторов. Для этого можно воспользоваться документацией SMLT, примерами кода и рекомендациями других пользователей.
	
	\textbf{Обучение и тестирование модели:}\\
	При использовании функций машинного обучения необходимо обучить модель на обучающей выборке и проверить ее на тестовой выборке. Для этого можно воспользоваться функциями \textit{fit} и \textit{predict}. Необходимо также выбирать подходящие параметры модели, такие как тип ядра, параметр регуляризации, глубина дерева и другие.
	
	\textbf{Оценка качества модели:}\\
	При использовании функций машинного обучения необходимо оценивать качество модели. Для этого можно воспользоваться функциями \textit{confusionmat}, \textit{roc}, \textit{lift} и другими. Необходимо также выбирать подходящие метрики качества, такие как точность, чувствительность, специфичность, F1-мера и другие.
	
	\textbf{Визуализация результатов:}\\
	При использовании функций SMLT необходимо визуализировать результаты анализа. Для этого можно воспользоваться функциями \textit{plot}, \textit{histogram}, \textit{boxplot}, \textit{scatter} и другими. Визуализация результатов помогает лучше понять данные, обнаружить закономерности и выявить ошибки.
	
	
	\section{Заключение}
	
	В этой справке были рассмотрены основные функции Statistics and Machine Learning Toolbox, которые являются мощными инструментами для обработки экспериментальных данных, постановки эксперимента и машинного обучения в MATLAB. Эти инструменты предоставляют широкий спектр функций и возможностей для решения типичных задач в этих областях. Знакомство с этими инструментами позволит эффективно решать задачи обработки экспериментальных данных, постановки эксперимента и машинного обучения в MATLAB.\\
	
	
	\section{Список литературы}
	
	1. The MathWorks, Inc. Statistics and Machine Learning Toolbox Documentation.\\
	2. The MathWorks, Inc. MATLAB Documentation.\\

		
	\end{titlepage}
	
	\end{document}