\documentclass[a4paper, 12pt]{article}%тип документа

%отступы
\usepackage[left=1.5cm,right=1cm,top=2cm,bottom=3cm,bindingoffset=0cm]{geometry}
\setlength{\parindent}{5ex}

%Русский язык
\usepackage[T2A]{fontenc} %кодировка
\usepackage[utf8]{inputenc} %кодировка исходного кода
\usepackage[english,russian]{babel} %локализация и переносы

%Вставка картинок
\usepackage{graphicx}
\graphicspath{{pictures/}}
\DeclareGraphicsExtensions{.pdf,.png,.jpg,}
\usepackage{wrapfig}

%Графики
\usepackage{pgfplots}
\pgfplotsset{compat=1.9}

%Математика
\usepackage{amsmath, amsfonts, amssymb, amsthm, mathtools}
\usepackage{derivative}

%Таблицы
\usepackage{longtable} 
\usepackage{float}

%Римские цифры
\newcommand{\RomanNumeralCaps}[1]{\uppercase\expandafter{\romannumeral#1}}

\usepackage{multirow}



\begin{document}
	
	
   \textbf {Вторая лемма Стрэнга.}\\
	Предположим, что $\left(a_h(\cdot, \cdot)\right)_{h \in \mathcal{H}}$ равномерно ограничена и равномерно $V_h$-эллиптическое семейство аппроксимированных билинейных форм и предположим что $u \in$ $V$ и $u_h \in V_h, h \in \mathcal{H}$, являются единственными решениями вариационных задач $a(u,v) = l(v), \hspace{4mm} v \in V$ и (5.88), соответственно.
	
	Тогда существует $C > 0$,  $C \in \mathbb{R}$, независимая от $h \in \mathcal{H}$, такая, что
	$$
	\left\|u-u_h\right\|_h \leq C\left(\inf _{v_h \in V_h}\left\|u-v_h\right\|_h+\sup _{w_h \in V_h} \frac{\left|a_h\left(u, w_h\right)-\ell\left(w_h\right)\right|}{\left\|w_h\right\|_h}\right) .
	$$
	
	\vspace{10mm}
	
	\textbf{Формальная оценка ошибки} - это оценка, которая основана на аналитических выражениях или аппроксимациях, использующих математические модели или методы, без использования экспериментальных данных или наблюдений. Этот тип оценок может быть полезен для оценки точности модели или алгоритма в теоретических или абстрактных сценариях, когда экспериментальные данные недоступны или необходимо провести анализ в широком диапазоне условий.
	
	Ограниченность семейства аппроксимирующих билинейных форм относится к вопросу о том, можно ли найти константу C, такую что для любой билинейной формы B(x, y) из данного семейства ошибка аппроксимации B(x, y) с помощью оператора Клемента не превосходит C.
	
	Семейство аппроксимирующих билинейных форм - это множество билинейных форм, которые можно аппроксимировать с помощью интерполяционного оператора, такого как оператор Клемента. Ограниченность семейства означает, что ошибка аппроксимации для любой билинейной формы из этого семейства имеет ограниченный размер, зависящий только от константы C, а не от конкретной билинейной формы или выбранного элемента.
	
	\vspace{10mm}
	
	\textbf{Формула интерполяционного оператора Клемента} для треугольника с вершинами A, B и C и значениями функции f(A), f(B) и f(C) в этих точках может быть записана следующим образом:
	
	$$L_C(x, y) = a_1 + a_2 x + a_3 y + a_4  x y + a_5 x^2 + a_6 y^2$$
	
	Коэффициенты $ a_1, a_2, a_3, a_4, a_5 и a_6 $ могут быть получены из симметричной матрицы, состоящей из полиномов, определяющих аппроксимацию функции внутри треугольника. Для этого можно использовать следующие соотношения:
	
	$$a_1 = f(A)$$
	$$a_2 = \frac{f(B) - f(C)}{(Bx - Cx)} $$
	$$a_3 = \frac{f(C) - f(A)}{(Cy - Ay)}$$
	$$a_4 = \frac{f(B) - f(A)}{(Bx  Cy - Ax  Cx)} $$
	$$a_5 = \frac{f(B) - f(A)}{2 (Bx^2 - Ax^2)} $$ 
	$$a_6 = \frac{f(C) - f(A)}{2(Cy^2 - Ay^2)} $$
	
	Здесь Bx и By - координаты точки B, Cx и Cy - координаты точки C, $a_1, a_2, a_3, a_4, a_5 и a_6 $ - коэффициенты оператора Клемента.
	
	Эта формула позволяет интерполировать функцию f(x, y) внутри треугольника с вершинами A, B и C, используя только значения функции в этих точках.
	Локально-аппроксимационные свойства оператора Клемента заключаются в том, что он позволяет аппроксимировать функцию внутри элемента с учетом только ее значений в вершинах элемента и без использования информации о соседних элементах. Это делает оператор Клемента эффективным инструментом для интерполяции функций на ограниченных областях, особенно когда область имеет сложную геометрию или разделена на неравномерно разбитые элементы.
	
\end{document}