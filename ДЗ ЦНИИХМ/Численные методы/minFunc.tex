\documentclass[a4paper, 12pt]{article}%тип документа

%отступы
\usepackage[left=1.5cm,right=1cm,top=2cm,bottom=3cm,bindingoffset=0cm]{geometry}
\setlength{\parindent}{5ex}

%Русский язык
\usepackage[T2A]{fontenc} %кодировка
\usepackage[utf8]{inputenc} %кодировка исходного кода
\usepackage[english,russian]{babel} %локализация и переносы

%Вставка картинок
\usepackage{graphicx}
\graphicspath{{pictures/}}
\DeclareGraphicsExtensions{.pdf,.png,.jpg,}
\usepackage{wrapfig}

%Графики
\usepackage{pgfplots}
\pgfplotsset{compat=1.9}

%Математика
\usepackage{amsmath, amsfonts, amssymb, amsthm, mathtools}
\usepackage{derivative}

%Таблицы
\usepackage{longtable} 
\usepackage{float}

%Римские цифры
\newcommand{\RomanNumeralCaps}[1]{\uppercase\expandafter{\romannumeral#1}}

\usepackage{multirow}



\begin{document}
	\begin{titlepage}
		\begin{center}
			\textsc{Федеральное государственное автономное образовательное учреждение высшего образования«Московский физико-технический институт (национальный исследовательский университет)»\\[5mm]
			}
			
			\vfill
			
			\textbf{Решение задачи минимизация функционала.
				\\[50mm]
			}
			
		\end{center}
		
		\hfill
		\begin{minipage}{.5\textwidth}
			Выполнили студенты:\\[2mm]
			Сериков Василий Романович\\[2mm]
			Сериков Алексей Романович\\[2mm]
			Данилов Иван Владимирович\\[2mm]
			группа: Б03-102\\[5mm]
			
		\end{minipage}
		\vfill
		\begin{center}
			Москва, 2023 г.
		\end{center}
		
	\end{titlepage}
	
	\newpage
	\setcounter{page}{2}
	
	\textbf{Цель работы: }\\
	
	Решить краевую задачу, описывающую стационарный процесс.\\
	
	\textbf{Теория: }\\
	
	Имеем краевую задачу:
	$$
	L u=-\operatorname{div}(k \operatorname{grad} u)+q u=\nabla \kappa \nabla u+q u=f
	$$
	
	С граничными условиями Неймана и Дирихле:
	$$
	\begin{gathered}
		\left.u\right|_{\Gamma_D}=\Phi \\
		-\vec{n} \cdot \kappa \nabla u+\left.\alpha u\right|_{\Gamma_\kappa}=\Psi
	\end{gathered}
	$$
	
	Где, заданная область $\Omega=(0 ; 1)$ х $(0 ; 1)$, $\mathrm{f}-$ непрерывна в $\Omega$, $\kappa=\kappa^T>0$, решение ищем, как непрерывную функцию в $C^2(\Omega)$.
	
	А для граничных условий выполняется следующие:
	$$
	\begin{aligned}
		& \Gamma_D=\overline{\Gamma_D} \subset \partial \Omega, \\
		& \Gamma_N=\partial \Omega-\Gamma_D ;
	\end{aligned}
	$$
	$\mathrm{L}$ - положительно определенный оператор, $L=L^*>0,(L u, v)=(u, L v)$, тогда определено скалярное произведение:
	$$
	[u, v]=(L u, v)=(u, L v),
	$$
	
	тогда для задачи (1) в области $\Omega$ верно следующие:
	$$
	(L u, v)=(f, v)
	$$
	
	Определим функционал J(u), как:
	$$
	J(u)=\frac{1}{2}(L u, u)-(f, u)
	$$
	
	Тогда верно следующие:
	$$
	\nabla J(u)=L u-f=0
	$$
	
	Таким образом, задача свелась к минимизации функционала J(u)

	%Краевая задача вида для уравнения эллиптического типа, при граничном условии третьего рода:
%	$$
%	\begin{gathered}
	%	L u=-\operatorname{div}(\kappa \operatorname{grad} u)+q u=-\nabla \kappa \nabla u+q u=f \\
%		\left.u\right|_{\Gamma_0}=\Phi \\
%		-\vec{n} \cdot \kappa \nabla u+\left.\alpha u\right|_{\Gamma_{\mathrm{n}}}=\Psi
%	\end{gathered}
%	$$
	 
	Домножаем обе части на u и переходим к интегрированию по заданной области:
	$$
	-\int_{\Omega} u \nabla \cdot \kappa \nabla u d x+\int_{\Omega} q u^2 d x
	$$
	
	При этом делаем следующую нормировку:
	$$
	|u|=\sqrt{(L u, u)}=[u, u]^{1 / 2}
	$$
	
	Тогда выполняется следующие:
	$$
	-\int_{\Omega} u \nabla \cdot \kappa \nabla u d x+\int_{\Omega} q u^2 d x=\int_{\Omega}|\nabla u|^2 d x
	$$
	
	После упрощения получим:
	$$
	\int_{\Omega}\left(\left|\kappa^{1 / 2} \nabla u\right|^2+q u^2\right) d x=\int_{\Omega}|\nabla u|^2 d x
	$$

	Если есть ограниченная область $\Omega$ с липшицевой границей и коэффициентом $a_{\alpha}$ в дифференциальном операторе второго порядка А из класса $C^{|\alpha|}(\bar{\Omega})$, то для произвольных функций $u, v$ из $C_2^1(\Omega)$ справедлива формула Грина:
	$$
	\int_{\Omega}\left(v A u-u A^T v\right) d x=\int_{\partial \Omega}\left(a_0\left(v \frac{\partial u}{\partial N}-u \frac{\partial v}{\partial N}\right)+a_* u v\right) d \Gamma,
	$$
	
	$$
	\begin{gathered}
		a_0=\left(\sum\left(\sum a^{i j} n_j\right)^2\right)^{1 / 2}, \\
		a_*=-\sum a^{i j} n_j,
	\end{gathered}
	$$
	
	Тогда, переходя к границе Неймана $\partial \Omega \rightarrow \Gamma_N$ :
	$$
	\int_{\Omega}|\nabla u|^2 d x=\int_{\Omega}\left(\left|\kappa^{1 / 2} \nabla u\right|^2+q u^2\right) d x+\int_{\Gamma_s} \alpha u^2 d s-\int_{\Gamma_\kappa} \Psi u d s
	$$


	1)Область $\bar{\Omega}=\Omega+\Gamma$ разбивают на $\mathrm{N}$ подобластей, называемых конечными элементами, так, что: $\Omega=\cup \Omega_k, \Gamma=\cup \Gamma_k, k \in[1, N]$.\\
	2)В каждом конечном элементе $\overline{\Omega_k}$, выбирается система узлов с нумерацией, в которых значения искомой функции ищутся.\\
	3)Каждому узлу приписывается базисная функция, такая что в этом узле она равна единице, а в остальных нумерованных узлах - нулю : $\varphi_k\left(x_{\mathrm{m}}\right)=\delta_{\mathrm{km}}$. Число базисных функций равно числу узлов.\\
	4)Решение задачи строится в виде линейных комбинаций базисных функций:
	$$
	u=\sum \alpha_k \varphi_k \text {. }
	$$\\
	5)Это решение подставляется в задачу, решением задачи является некоторая функциональная невязка.\\
	6)Функциональная невязка минимизируется\\
	
	\textbf{Решение: }\\
	Имеем краевую задачу: \\
	$$ Lu = - \Delta u + qu, \hspace{4mm} q(x, y) > 0 $$
	
	И краевые условия: \\
	
	$$ u(0,0) = \phi_0$$
	
	$$ \pdv{u(1,1) }{n} + \alpha u = \phi_1$$
	
	Тогда функционал $J(u): $\\
	
	$$ J(u) = \int_{\Omega} (-\Delta u\cdot u + qu^2 )dxdy = \int_{\Omega} ((\nabla u)^2 + qu^2)dxdy -\nabla u \cdot u \Bigr|_{(0,0)}^{(1,1)}$$ 
	
	Введем дискретизацию:\\
	
	$$ u^h = \Sigma_{i=1}^{n}\Sigma_{j=1}^{n} u_{ij}^h \varphi_{ij}(x,y)$$  
	
	Тогда: \\
	
	$$ \nabla J(u^h) = \Sigma_{i=1}^{n} \Sigma_{i=1}^{n}u_{ij}^h \Delta \varphi_{ij} + q\Sigma_{i=1}^{n}\varphi_{ij}^2 u_{ij}^{h2} - \int_{\Omega} \Sigma_{i=1}^{n} f(x,y) \varphi_{ij}$$
	
	\textbf{Вывод: }\\
	
	В ходе работы мы познакомились с методом конечных элементов для решения краевой задачи и попробовали решить двумерную задачу. 
	
	
	\end{document}